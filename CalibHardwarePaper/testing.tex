\section{Testing, Cleaning and Commissioning} \label{sec:Testing}\label{sec:Commissioning}
CALIS has been assembled at FNAL from components produced at FNAL and University of Hawaii. After initial basic functionality tests at FNAL, CALIS was shipped pre-assembled to LNGS, where it underwent a comprehensive testing and calibration program. While still outside clean room CRH, CALIS was installed on a high bay platform in the LNGS underground hall C and tested at its full length. Besides testing basic Z-motion, articulation and XY-rotation, all details of CALIS operations ranging from testing functionality of motor controls, high limit and articulation switch to recovery scenarios after e.g.~power failures during deployment were validated.

An important aspect was source's Z-position calibration as a function of cable length before and after source arm articulation, which is a nonlinear function of motor step counts, as mentioned in Sec.~\ref{sec:Nonlinearity:MotorStepCounts}. Furthermore source's XY- and Z-position accuracy and precision was estimated.

This testing campaign's results were reviewed by an internal review board and approval for installation inside CRH was granted. CALIS was cleaned according to official cleaning procedures and installed on gate valve in September 2014.
After installation on gate valve a testing focus was system's light and helium leak tightness, and nitrogen and vacuum systems testing, as these could only be tested fully after installation. A more detailed description of tests performed at FNAL and LNGS can be found in \cite{thesis:Hackett, thesis:Edkins}.

\subsection*{XY- and Z-position}
Tests in air and in \lsv's scintillator revealed source position accuracy and precision is dominated by uncertainties during articulation. Positioning in Z before articulation is highly accurate and precise: deployment speed is very low, barely visible to the naked eye (4 mm/s), which minimizes lateral motion during deployment and contact with housing or organ pipe is avoided during deployment. Yet during articulation a swing in XY arises from tiny laterally imbalanced forces originating in articulating cable pull. 

To ensure deployment precision a procedure has been worked out to make reliably gentle contact with the cryostat, thereby eliminating precision uncertainty in XY: After positioning deployment device in Z, source arm is articulated to horizontal while it is pointing away from the cryostat. Only then source is brought into contact with cryostat through a XY-rotation while monitoring photomultiplier tube (PMT) scaler rates, which increase while source is approaching, yet plateaus as soon as contact with the cryostat is made, even if XY-rotation continues. This provides a reliable XY- and Z-position for calibration source and was used throughout calibration campaigns.