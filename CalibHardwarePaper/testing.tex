\section{Testing, Cleaning and Commissioning} \label{sec:Testing}\label{sec:Commissioning}
CALIS was assembled at FNAL from components produced at FNAL and at the University of Hawaii. After initial basic functionality tests at FNAL, CALIS was shipped pre-assembled to LNGS, where it underwent a comprehensive testing and calibration program. While still outside the clean room CRH, CALIS was installed on a high bay platform in the LNGS underground Hall C and tested to its full deployment length. Besides testing basic $z$-motion, articulation and $xy$-rotation, all details of CALIS operations were tested. These included the testing of the functionality of motor controls, the high limit and articulation switch. Validation was concluded with recovery scenarios after, e.g.,~power failures during deployment.

An important aspect was $z$-position calibration of the source as a function of cable length before and after source arm articulation, which is a nonlinear function of motor step counts, as mentioned in Sec.~\ref{sec:Nonlinearity:MotorStepCounts}. Furthermore, the source's $xy$- and $z$-position accuracy and precision were estimated.

This testing campaign's results were reviewed by an internal review board and approval for installation of CALIS inside CRH was granted. CALIS was cleaned according to official cleaning procedures and installed on its gate valve in September 2014.
After installation on the gate valve a testing focus was the system's light and helium leak tightness, as well as validation of nitrogen and vacuum systems, as these could only be tested fully after installation. A more detailed description of tests performed at FNAL and LNGS can be found in \cite{thesis:Hackett, thesis:Edkins}.

\subsection*{$xy$- and $z$-position}
Tests in air and in the LSV's scintillator revealed that the dominant source of uncertainty in the source position arises during articulation. Positioning in $z$ before articulation is highly accurate and precise: deployment speed is very low, barely visible to the naked eye (0.4 cm/s), which minimizes lateral motion and avoids contact with the housing or the organ pipe during deployment. Yet during articulation a swing of the source in $xy$ arises from tiny laterally imbalanced forces originating in the cable manipulated for articulation. 

To ensure deployment precision, a procedure has been worked out to make reliably gentle contact with the cryostat, thereby eliminating precision uncertainty in $xy$: After positioning the deployment device in $z$, the source arm is articulated to horizontal while it is still pointing away from the cryostat. Only then is the source brought into contact with the cryostat through an $xy$-rotation of the upper assembly, while monitoring scaler rates in the TPC's photomultiplier tubes (PMT).  These increase while the source is approaching, yet plateau as soon as contact with the cryostat is made, even if $xy$-rotation continues. This procedure provides a reliable $xy$- and $z$-position for the calibration source and was used throughout all calibration campaigns.