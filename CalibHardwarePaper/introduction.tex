%---
\section{Introduction}\label{sec:intro}\label{sec:introduction}

\dsf\ is a Liquid Argon Time Projection Chamber (\lar\ \tpc), operated in Italy's Gran Sasso National Laboratory (LNGS) to search for nuclear recoils induced by weakly interacting massive particles (WIMPs). The first physics result was reported in \cite{Agnes:2015gu} based on 50 live data collection days with Atmospheric Argon (AAr), providing the most sensitive limit on a dark matter search using a \lar\ \tpc\ to date with a 90\% CL upper limit on the WIMP-nucleon spin-independent cross section of $6.1 x 10^{-44}$ cm$^2$ for a WIMP mass of 100 GeV/c$^2$.  %along with two other key results: ar bg can be suppressed for ton-scale experiments using \uar and efficiency of the veto 

A first WIMP search using argon extracted from underground sources (Underground Argon, UAr) has been reported in \cite{Agnes:2015_uar}, following the WIMP search with AAr. UAr has a lower concentration of the radioactive $\beta$-emitter $^{39}$Ar by a factor (1.4 $\pm$ 0.2) $\times\, 10^3$ relative to AAr. Calibration campaigns have been performed in the presence of AAr and UAr.

\dsf\ apparatus is described in detail in \cite{Agnes:2015gu}. As shown in fig.~\ref{fig:wholeAssembly_insideDetectors} it features a \lar\ \tpc\ surrounded by a 30 t liquid scintillator-based veto (LSV) system, placed inside a water Cerenkov veto detector (\wcv), both which measure in-situ and suppress radiogenic and cosmogenic backgrounds. On \wcv's top is a radon-free clean room (CRH) housing the cryogenic supply system and electronics (Fig.~\ref{fig:CALIS_photos}). The \lsv's inside is accessible from CRH through four access ports called organ pipes closed by gate valves. 

%After this introduction the CALIS design requirements and hardware realization are described in Sec.~\ref{sec:hardware}. Calibration campaigns and some of their physics highlights are discussed in Sec.~\ref{sec:CalibCampaigns}, before concluding in Sec.~\ref{sec:Conclusion}.

\begin{figure}[htbp]
 \centering
\includegraphics[width=\textwidth]{Figures/DS50_with_CALIS}
\caption{A conceptual drawing of CALIS (1) installed in radon-free clean room CRH (2) atop water Cerenkov veto (\wcv, 3) and with deployment device (4) containing the source deployed in liquid scintillator veto (LSV, 5) next to liquid argon time projection chamber's (\lar\ \tpc) cryostat (6). Clean room and LSV are connected through four access ports called organ pipes (only one of which is drawn in above sketch: (7)). All four organ pipes end in CRH at gate valves (8) which can be manually opened or closed. During normal operations all four organ pipes are closed. Not included in sketch are tubes connecting cryogenic systems in CRH to cryostat in \lsv\ \cite{Agnes:2015qyz}.\label{fig:wholeAssembly_insideDetectors}\label{fig:DS50_with_CALIS}}
\end{figure}

\begin{figure}[htbp]
 \centering
\subfigure{\includegraphics[width=0.335\textwidth]{./Figures/CALISinCRH_PetersComment.png}}
\subfigure{\includegraphics[width=0.52\textwidth]{./Figures/Next2Cryostat_80cm.png}}
\caption{\textit{left}: CALIS after installation inside radon-free clean room CRH. The organ pipe is 80 cm off-center with respect to TPC's vertical z-axis. \textit{right}: Photograph taken with a camera looking upwards into \lsv\ from the bottom. It shows source deployment device deployed through one of the organ pipes visible on top right. The arm is articulated and source is next to \lar\ \tpc's cryostat \cite{Agnes:2015qyz}.
\label{fig:CALIS_photos}}
\end{figure}

\subsection*{Laser calibration}
\wcv, \lsv\ and \tpc\ electronics gain calibration is performed with dedicated Laser systems, in place in each of three subdetectors. \wcv\ laser calibration is sufficient to veto muons (and their secondaries) with high efficiency. \tpc\ detector response has been calibrated using the internal ${39}$Ar and $83m$Kr, that has been added into \lar\ recirculation system during dedicated calibration campaigns \cite{Agnes:2015gu}.



%time line plot, how did the veto change during the calibration campaigns - allowing for systematics studies.
%table on what was the goal of each campaign. What sources were used


%safety requirements

%design section
%installation: before, after

