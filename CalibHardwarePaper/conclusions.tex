\section{Conclusions}\label{sec:Conclusions}\label{sec:Conclusion}
CALIS is a simple, affordable and effective source deployment system that has been successfully used to deploy sources in the LSV and next to the TPC and to conduct several successful calibration campaigns.%No adverse effects on the LSV or TPC have been noticed.

%summarize that the LSV and TPC detector have not been negatively affected.
%that's a sentence for the summary: The critical contribution by CALIS is the neutron veto detection efficiency calibration using dedicated neutron sources
%neutron gun inside a dedicated deployment device currently under development (Section \ref{sec:Outlook}).
%Refer to the next generation DS-detector.
%\section{Epilogue}\label{sec:Epilogue}
%here I would list all upcoming DarkSide papers and how they are in relationship with each other. Since so many papers are in preparation I would find that %helpful. Even if it is not eventually put into the paper.

%%%%%%%%%%%%%%%%%%%%%%%%%%%%%%%%%%%%%
%%%%%%%%%%%%%%%%%%%%%%%%%%%%%%%%%%%%%

\section{Acknowledgements}\label{sec:Acknowledgements}
The DarkSide-50 Collaboration would like to thank LNGS laboratory and its staff for invaluable technical and logistical support. This report is based upon work supported by the US NSF (Grants PHY-0919363, PHY-1004072, PHY-1004054, PHY-1242585, PHY-1314483, PHY-1314507 and associated collaborative grants; grants PHY-1211308 and PHY-1455351), the Italian Istituto Nazionale di Fisica Nucleare (INFN), the US DOE (Contract Nos. DE-FG02-91ER40671 and DE-AC02-07CH11359), the Russian RSF (Grant No 16-12-10369), and the Polish NCN (Grant UMO-2014/15/B/ST2/02561). %UMO-2012/05/E/ST2/02333
We thank the staff of the Fermilab Particle Physics, Scientific and Core Computing Divisions for their support. We acknowledge the financial support from the UnivEarthS Labex program of Sorbonne Paris Cit\'{e} (ANR-10-LABX-0023 and ANR-11-IDEX-0005-02) and from the S\~{a}o Paulo Research Foundation (FAPESP).