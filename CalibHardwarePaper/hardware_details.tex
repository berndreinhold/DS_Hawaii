\subsection{Hardware Details and Safety Features}\label{sec:HardwareDetails}\label{sec:SafetyFeatures}
CALIS offers various safety features ensuring that this device runs smoothly, that no components are lost inside the detector, that any detector contamination by dirty or incompatible materials is avoided, that pressure is maintained and that the introduction of oxygen or water into the LS and TMB is avoided. %, operation in the volume that excludes possibility of contact with PMTs or light pulsers (pacman) attached to each PMT.

\begin{description}

\item[Cable strength:]
The cables holding the deployment device are rated for loads over 1300\,lbs, while the device's weight is at the level of 20-30\,lbs, thus well below the cables' breaking strength. \mymarginpar{lbs or kg or are both OK?}

\item[Drive mechanism:]
If a power failure occurs, the magnetic break ensures that the deployment device does not move. Servo motor torque is limited in case of an unexpected load and the risk of breaking the cable is avoided.

The speed reducer is implemented in a double worm gear design. The primary worm gear has a 50:1 reduction and the secondary worm has a 82:1 reduction. The servo motor input speed is 2400 RPMs, the output is 0.6 RPM and its weight capacity is 148 lbs. If a power failure occurs, the speed reducer can hold the load at any position without back drive. The motor speed has been limited to 0.4\,cm/s minimizing any lateral oscillation of deployment device during lowering and raising the source. This is also the maximum speed at which the motor does not overheat.

\item[Manual retraction system:]
It is possible to manually retract the deployment device back to its home position and to close the gate valve in the unlikely event of a complete motor failure while the deployment device is deployed. The motor is disengaged, and a wrench is used to manually wind the cables back on the spools and to retract the deployment device back above gate valve. During this procedure the nitrogen blanket protecting the \lsv\ is preserved. 
   
\item[High limit switch:]
A high limit switch is a hardware interlock, that prevents the deployment device hitting cable spools and gears, should it pass beyond the home position in the CALIS housing (Fig.~\ref{fig:CALISMechanism}). 

\textit{Neither the manual retraction system had to be used so far nor has the high limit switch been activated during calibration campaigns.}
    
\item[Light and leak tightness of CALIS:]
When the deployment device is next to the cryostat, the gate valve is open and data is also taken with the \lsv. A requirement is therefore an absolute light tight and pressure leak tight housing. All view ports are covered with light tight covers when the gate valve is open. Both light and leak tightness was extensively validated throughout the manufacturing process, commissioning and during calibration campaigns (Sec.~\ref{sec:Commissioning}).

\item[Securing source:] 
All connection points for the source and the source arm have been secured with two push locking pins that cannot be disengaged without a person pressing the pin. In addition, the source holder and its two locking pins are all tethered from outside the view port until they are locked in place eliminating the possibility of accidentally falling into the interior of the CALIS housing.

\end{description}
	
%%%%%%%%%%%%%%%%%%%%%%%%%%%%%%%%%%%%%%%%%
%%%%%%%%%%%%%%%%%%%%%%%%%%%%%%%%%%%%%%%%%