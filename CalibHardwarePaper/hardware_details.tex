\subsection{Hardware Details and Safety Features}\label{sec:HardwareDetails}\label{sec:SafetyFeatures}
CALIS offers various safety features ensuring that this device runs smoothly, no components are lost inside detector, avoid any detector contamination by dirty or incompatible materials, maintain pressure and avoiding introducing oxygen or water in contact with LS and TMB. %, operation in the volume that excludes possibility of contact with PMTs or light pulsers (pacman) attached to each PMT.

\begin{description}

\item[Cable strength:]
Deployment device holding cables are rated for loads over 1300\,lbs, while deployment device weight is at the level of 20-30\,lbs, thus well below cable's breaking strength. 

\item[Drive mechanism:]
If a power failure occurs, the magnetic break ensures deployment device does not move. Servo motor torque is limited in case of an unexpected load and the risk of breaking the cable is avoided.

Speed reducer (gears) is a double worm gear design. Primary worm gear has a 50:1 reduction and secondary worm has a 82:1 reduction. Servo motor input speed is 2400 RPMs, output is 0.6 RPM and weight capacity is 148 lbs. If a power failure occurs speed reducer can hold load at any position without back drive. Motor speed has been limited to 0.4\,cm/s minimizing any lateral oscillation of deployment device during lowering and raising the source. This is maximum speed at which motor does not overheat.

\item[Manual retraction system:]
It is possible to manually retract deployment device back to its home position and close gate valve in unlikely case of complete motor failure while source is deployed. Motor is disengaged, and a wrench is used to manually wind cable back on spools and retract deployment device back above gate valve. 
   
\item[High limit switch:]
A high limit switch is a hardware interlock, that prevents deployment device hitting cable spools and gears, should it pass beyond home position in CALIS housing (Fig.~\ref{fig:CALISMechanism}). 

\textit{Neither manual retraction system had to be used so far nor has high limit switch been activated during calibration campaigns.}
    
\item[Light and leak tightness of CALIS:]
When deployment device is next to cryostat gate valve is open and data is also taken with \lsv. A requirement is an absolute light tight and pressure leak tight housing. All view ports are covered with light tight covers when organ pipe gate valve is open. Both light and leak tightness was extensively validated throughout manufacturing process, commissioning and during calibration campaigns (Sec.~\ref{sec:Commissioning}).

\item[Securing source:] 
All connection points for source and arm have been secured with two push locking pins that cannot be disengaged without a person pressing the pin. In addition, source holder and two locking pins are all tethered from outside view port until they are locked in place eliminating possibility of accidentally falling into CALIS housing's interior.

\end{description}
	
%%%%%%%%%%%%%%%%%%%%%%%%%%%%%%%%%%%%%%%%%
%%%%%%%%%%%%%%%%%%%%%%%%%%%%%%%%%%%%%%%%%